\documentclass[addpoints,12pt]{exam}
\usepackage{amsmath, amssymb}
\linespread{1.1}
\usepackage{hyperref}
\usepackage{enumerate}
\usepackage{multirow}

%\printanswers
\noprintanswers

\title{Review Questions}
\author{Econ 103}
\date{Spring 2018}

\begin{document}
\maketitle

\section*{About This Document}
These questions are the ``bread and butter'' of Econ 103: they cover the basic knowledge that you will need to acquire this semester to pass the course. 
There are between 10 and 15 questions for each lecture.
After a given lecture, and before the next one, you should solve all of the associated review questions.
To give you an incentive to keep up with the course material, all quiz questions for the course will be randomly selected from this list.
For example Quiz \#1, which covers lectures 1--2, will consist of one question drawn at random from questions 1--10 and another drawn at random from questions 12--24 below.
We will not circulate solutions to review questions.
Compiling your own solutions is an important part of studying for the course.
We will be happy to discuss any of the review questions with you in office hours or on Piazza, and you are most welcome to discuss them with your fellow classmates.
Be warned, however, that merely memorizing answers written by a classmate is a risky strategy.
It may get you through the quiz, but will leave you woefully unprepared for the exams.
There is no curve in this course: to pass the exams you will have to learn the material covered in these questions.
Rote memorization will not suffice.

%%%%%%%%%%%%%%%%%%%%%%%%%%%%%%%%%%%%%%%%%%%%%%%%%%%%%%%%%%%%%%%%%%
\section*{Lecture \#1 -- Introduction}
\begin{questions}

  \question Define the following terms and give a simple example: \emph{population}, \emph{sample}, \emph{sample size}.
  \question Explain the distinction between a \emph{parameter} and a \emph{statistic}.
  \question Briefly compare and contrast \emph{sampling} and \emph{non-sampling} error.
  \question Define a \emph{simple random sample}. Does it help us to address sampling error, non-sampling error, both, or neither? 

\question A drive-time radio show frequently holds call-in polls during the evening rush hour. Do you expect that results based on such a poll will be biased? Why? 
	\begin{solution}
    They will likely be biased.
		People who are listening to the radio during rush hour are disproportionately likely to be commuters driving home from work. People who are employed and drive to work are not representative of the population at large.  
	\end{solution}


\question Dylan polled a random sample of 100 college students. In total 20 of them said that they approved of President Trump. Calculate the margin of error for this poll.
\begin{solution}
  $2 \sqrt{P(1-P)/n} = 2 \sqrt{0.2 \times 0.8 / 100} = 0.08$
\end{solution}

\question Define the term \emph{confounder} and give an example.

\question What is a randomized, double-blind experiment? In what sense is it a ``gold standard?'' 

	
\question Indicate whether each of the following involves experimental or observational data.
	\begin{parts}
		\part A biologist examines fish in a river to determine the proportion that show signs of disease due to pollutants poured into the river upstream.
		\begin{solution}
		Observational
		\end{solution}
		\part In a pilot phase of a fund-raising campaign, a university randomly contacts half of a group of alumni by phone and the other half by a personal letter to determine which method results in higher contributions.
				\begin{solution}
				Experimental
		\end{solution}
		\part To analyze possible problems from the by-products of gas combustion, people with with respiratory problems are matched by age and sex to people without respiratory problems and then asked whether or not they cook on a gas stove.
				\begin{solution}
				Observational
		\end{solution}
		\part An industrial pump manufacturer monitors warranty claims and surveys customers to assess the failure rate of its pumps.
				\begin{solution}
				Observational
		\end{solution}
	\end{parts}


  \question Based on information from an observational dataset, Amy finds that students who attend an SAT prep class score, on average, 100 points better on the exam than students who do not. In this example, what would be required for a variable to \emph{confound} the relationship between SAT prep classes and exam performance? What are some possible confounders?
\begin{solution}
\end{solution}


%%%%%%%%%%%%%%%%%%%%%%%%%%%%%%%%%%%%%%%%%%%%%%%%%%%%%%%%%%%%%%%%%%
\fullwidth{\section*{Lecture \#2 -- Summary Statistics I}}

\question For each variable indicate whether it is nominal, ordinal, or numeric.
	\begin{parts}
		\part Grade of meat: prime, choice, good.
			\begin{solution}
				ordinal
			\end{solution}
		\part Type of house: split-level, ranch, colonial, other.
			\begin{solution}
				nominal
			\end{solution}
		\part Income
			\begin{solution}
			 numeric
			\end{solution}
	\end{parts}
	
\question Explain the difference between a histogram and a barchart.

\question Define \emph{oversmoothing} and \emph{undersmoothing}.

\question What is an \emph{outlier}?

\question Write down the formula for the sample mean. What does it measure? Compare and contrast it with the sample median. 

\question Two hundred students took Dr.\ Evil's final exam. The third quartile of exam scores was 85. Approximately how many students scored \emph{no higher} than 85 on the exam? 

\question Define \emph{range} and \emph{interquartile range}. What do they measure and how do they differ? 

\question What is a boxplot? What information does it depict?

\question Write down the formula for variance and standard deviation. What do these measure? How do they differ?


\question Suppose that $x_i$ is measured in inches. 
What are the units of the following quantities? 
	\begin{parts}
    \part Sample mean of $x$ 
    \begin{solution}
      inches
    \end{solution}
    \part Range of $x$
    \begin{solution}
      inches
    \end{solution}
		\part Interquartile Range of $x$
			\begin{solution}
        inches
	\end{solution}
		\part Variance of $x$
		\begin{solution}
		square inches 
		\end{solution}
    \part Standard deviation of $x$
    \begin{solution}
      inches
    \end{solution}
	\end{parts}

\question Evaluate the following sums:
\begin{parts}
  \part $\displaystyle\sum_{n=1}^3 n^2$
  \begin{solution}
    $\displaystyle\sum_{n=1}^3 n^2 = 1^2 + 2^2 + 3^2 = 1 + 4 + 9 = 14$
  \end{solution}
  \part $\displaystyle\sum_{n=1}^3 2^n$
  \begin{solution}
    $\displaystyle\sum_{n=1}^3 2^n = 2^1 + 2^2 + 2^3 = 2 + 4 + 8 = 14$
  \end{solution}
  \part $\displaystyle\sum_{n=1}^3 x^n$
  \begin{solution}
    $\displaystyle\sum_{n=1}^3 x^n = x + x^2 + x^3$
  \end{solution}
\end{parts}

\question Evaluate the following sums:
\begin{parts}
  \part $\displaystyle\sum_{k=0}^2 (2k + 1)$
  \begin{solution}
    $\displaystyle\sum_{k=0}^2 (2k + 1) = (2 \times 0 + 1) + (2 \times 1 + 1) + (2 \times 2 + 1) = 9$
  \end{solution}
  \part $\displaystyle\sum_{k=0}^3 (2k + 1)$
  \begin{solution}
    $\displaystyle\sum_{k=0}^3 (2k + 1) = \left[\sum_{k=0}^2 (2k + 1)\right] + (2 \times 3 + 1) = 9 + 7 = 16$
  \end{solution}
  \part $\displaystyle\sum_{k=0}^4 (2k + 1)$
  \begin{solution}
    $\displaystyle\sum_{k=0}^4 (2k + 1) = \left[ \sum_{k=0}^3 (2k + 1)\right] + (2 \times 4 + 1) = 16 + 9 = 25$ 
  \end{solution}
\end{parts}

\question Evaluate the following sums:
\begin{parts}
  \part $\displaystyle\sum_{i=1}^3 (i^2 + i)$
  \begin{solution}
  $\displaystyle\sum_{i=1}^3 (i^2 + i) = (1^2 + 1) + (2^2 + 2) + (3^2 + 3) = 20$
  \end{solution}
  \part $\displaystyle\sum_{n =-2}^2 (n^2 - 4)$
  \begin{solution}
    $\displaystyle\sum_{n =-2}^2 (n^2 - 4) = \left[(-2)^2 + (-1)^2 + (0)^2 + (1)^2 + (2)^2 \right] - (4 \times 5) = -10$
  \end{solution}
  \part $\displaystyle\sum_{n = 100}^{102} n$
  \begin{solution}
  $\displaystyle\sum_{n = 100}^{102} n = 100 + 101 + 102 = 303$
  \end{solution}
  \part $\displaystyle\sum_{n = 0}^2 (n + 100)$
  \begin{solution}
  $\displaystyle\sum_{n = 0}^2 (n + 100) = (0 + 1 + 2) + 3 \times 100 = 303$
  \end{solution}
\end{parts}

\question Express each of the following using $\Sigma$ notation:
  \begin{parts}
    \part $z_1 + z_2 + \cdots + z_{23}$
    \begin{solution}
      $\displaystyle \sum_{i=1}^{23} z_i$ 
    \end{solution}
    \part $x_1 y_1 + x_2 y_2 + \cdots + x_8 y_8$
    \begin{solution}
      $\displaystyle \sum_{i=1}^8 x_i y_i$
    \end{solution}
    \part $(x_1 - y_1) + (x_2 - y_2) + \cdots + (x_m - y_m)$
    \begin{solution}
      $\displaystyle \sum_{i=1}^m (x_i - y_i)$
    \end{solution}
    \part $x_1^3 f_1 + x_2^3 f_2 + \cdots + x_9^3 f_9$
    \begin{solution}
      $\displaystyle \sum_{i=1}^9 x_i^3 f_i$
    \end{solution}
\end{parts}


%%%%%%%%%%%%%%%%%%%%%%%%%%%%%%%%%%%%%%%%%%%%%%%%%%%%%%%%%%%%%%%%%%
\fullwidth{\section*{Lecture \#3 -- Summary Statistics II}}


\question Show that $\displaystyle \sum_{i = m}^n (a_i + b_i) = \sum_{i=m}^n a_i + \sum_{i=m}^n b_i$. Explain your reasoning.

\question Show that if $c$ is a constant then $\displaystyle \sum_{i=m}^n c x_i = c \sum_{i=m}^n x_i$. Explain your reasoning.

\question Show that if $c$ is a constant then $\displaystyle \sum_{i=1}^n c = cn$. Explain your reasoning.

\question Mark each of the following statements as True or False. You do not need to show your work if this question appears on a quiz, although you should make sure you understand the reasoning behind each of your answers.
	\begin{parts}
		\part $\displaystyle\sum_{i=1}^n (x_i/n) = \left(\sum_{i=1}^n x_i\right)/n$ 
		\part $\displaystyle\sum_{k = 1}^n x_k z_k = z_k \sum_{k = 1}^n x_k$ 
		\part $\displaystyle\sum_{k=1}^m x_k y_k = \left(\sum_{k=1}^m x_k\right) \left(\sum_{k=1}^m y_k\right)$ 
		\part $\displaystyle\left(\sum_{i=1}^n x_i \right)\left(\sum_{j=1}^m y_j\right) = \sum_{i=1}^n \sum_{j=1}^m x_iy_j$ 
		\part $\displaystyle\left(\sum_{i=1}^n x_i\right)/\left(\sum_{i=1}^n z_i\right)= \sum_{i=1}^n \left(x_i/z_i\right) $
	\end{parts}

  \question Show that $\sum_{i=1}^n (x_i - \bar{x}) = 0$. Justify all of the steps you use.

  \question Re-write the formula for skewness in terms of the z-scores $z_i = (x_i - \bar{x})/s$. Use this to explain the original formula: why does it involve a cubic and why does it divide by $s^3$?
  \begin{solution}
    \[
      \frac{1}{n}\frac{\sum_{i=1}^n (x_i - \bar{x})^3}{s^3} = \frac{1}{n} \sum_{i=1}^n \left( \frac{x_i - \bar{x}}{s} \right)^3 = \frac{1}{n} \sum_{i=1}^n z_i^3
    \]
  \end{solution}


\question How do we interpret the sign of skewness, and what is the ``rule of thumb'' that relates skewness, the mean, and median?

\question What is the distinction between $\mu, \sigma^2, \sigma$ and $\bar{x}, s^2, s$? Which corresponds to which?

\question What is the empirical rule?

\question Define \emph{centering}, \emph{standardizing}, and \emph{z-score}.

\question What is the sample mean $\bar{z}$ of the z-scores $z_1, \dots, z_n$? Prove your answer.

\question What is the sample variance $s_z^2$ of the z-scores $z_1, \dots, z_n$? Prove your answer.

\question Suppose that $-c < (a - x)/b < c$ where $b>0$. Find a lower bound $L$ and an upper bound $U$ such that $L < x < U$.
			\begin{solution}
				Rearranging, 
					$$-bc - a < -x < bc - a$$
				and multiplying through by $-1$,
					$$a - bc < x <a + bc$$
        \end{solution}

\question Compare and contrast \emph{covariance} and \emph{correlation}. Provide the formula for each, explain the units, the interpretation, etc.

\question Suppose that $x_i$ is measured in centimeters and $y_i$ is measured in feet. What are the units of the following quantities? 
	\begin{parts}
		\part Covariance between $x$ and $y$			
		\begin{solution}
	 centimeters $\times$ feet
	\end{solution}
		\part Correlation between $x$ and $y$
		\begin{solution}
		unitless
		\end{solution}
		\part Skewness of $x$
		\begin{solution}
		unitless
		\end{solution}
    \part $(x_i - \bar{x}) / s_x$
    \begin{solution}
      unitless
    \end{solution}
	\end{parts}

%%%%%%%%%%%%%%%%%%%%%%%%%%%%%%%%%%%%%%%%%%%%%%%%%%%%%%%%%%%%%%%%%%
\fullwidth{\section*{Lecture \#4 -- Regression I}}

\question In a regression using height (measured in inches) to predict handspan (measured in centimeters) we obtained $a = 5$ and $b = 0.2$. 
\begin{parts}
  \part What are the units of $a$?
  \part What are the units of $b$?
  \part What handspan would we predict for someone who is 6 feet tall?
\end{parts}

\question Plot the following dataset and calculate the corresponding regression slope and intercept \emph{without} using the regression formulas.\\ 
\begin{tabular}[h]{cc}
  $x$ & $y$\\
  \hline
   0 & 2\\
   1 & 1\\
   1 & 2
\end{tabular}

\question Write down the optimization problem that linear regression solves.

\question Prove that the regression line goes through the means of the data.

\question By substituting $a = \bar{y} - b\bar{x}$ into the linear regression objective function, derive the formula for $b$.

\question Consider the regression $\widehat{y} = a + bx$.
\begin{parts}
  \part Express $b$ in terms of the sample covariance between $x$ and $y$.
  \part Express the sample correlation between $x$ and $y$ in terms of $b$.
\end{parts}

\question What value of $a$ minimizes $\displaystyle\sum_{i=1}^n (y_i - a)^2$? Prove your answer.

\question Suppose that $s_{xy} = 30$, $s_x = 10$, $s_{y} = 6$, $\bar{y} = 12$, and $\bar{x} = 4$. Calculate $a$ and $b$ in the regression $\widehat{y} = a + bx$.
\begin{solution}
  \begin{align*}
  b &= s_{xy}/s_x^2 = 30 / 10^2 = 30/100 = 0.3\\
  a &= \bar{y} - b \bar{x} = 12 - 0.3 \times 4 = 12 - 1.2 = 10.8
  \end{align*}
\end{solution}

\question Suppose that $s_{xy} = 30$, $s_x = 10$, $s_{y} = 6$, $\bar{y} = 12$, and $\bar{x} = 4$. Calculate $c$ and $d$ in the regression $\widehat{x} = c + dy$. Note: we are using $y$ to predict $x$ in this regression!
\begin{solution}
  \begin{align*}
  b &= s_{xy}/s_y^2 = 30 / 6^2 = 30/36 = 5/6 \approx 0.83\\ 
  a &= \bar{y} - b \bar{x} = 12 - 5/6 \times 4 = 12 - 10/3 = 26/3 \approx 8.7
  \end{align*}
\end{solution}

\question A large number of students took two midterm exams. The standard deviation of scores on midterm \#1 was 16 points, while the standard deviation of scores midterm \#2 was 17 points. The covariance of the scores on the two exams was 124 points squared. Linus scored 60 points on midterm \#1 while Lucy scored 80 points. How much higher would we predict that Lucy's score on the midterm \#2 will be?

\question Suppose that the correlation between scores on midterm \#1 and midterm \#2 in Econ 103 is approximately 0.5. If the regression slope when using scores on midterm \#1 to predict those on midterm \#2 is approximately 1.5, which exam had the larger \emph{spread} in scores? How much larger?

%%%%%%%%%%%%%%%%%%%%%%%%%%%%%%%%%%%%%%%%%%%%%%%%%%%%%%%%%%%%%%%%%%
\fullwidth{\section*{Lecture \#5 -- Basic Probability I}}

\question What is the definition of probability that we will adopt in Econ 103?

\question Define the following terms:
\begin{parts}
  \part \emph{random experiment}
  \part \emph{basic outcomes}
  \part \emph{sample space}
  \part \emph{event}
\end{parts}


\question Define the following terms and give an example of each:
\begin{parts}
  \part \emph{mutually exclusive events}
  \part \emph{collectively exhaustive events}
\end{parts}

\question Suppose that $S = \left\{1, 2, 3, 4, 5, 6 \right\}$, $A = \left\{2, 3 \right\}$, $B = \left\{ 3, 4, 6 \right\}$, and $C = \left\{ 1, 5 \right\}$.
\begin{parts}
  \part What is $A^c$? 
  \part What is $A\cup B$?
  \part What is $A \cap B$?
  \part What is $A \cap C$?
  \part Are $A,B,C$ mutually exclusive? Are they collectively exhaustive?
\end{parts}

\question A family has three children. Let $A$ be the event that they have less than two girls and $B$ be the event that they have exactly two girls. 
\begin{parts}
  \part List all of the basic outcomes in $A$.
  \part List all of the basic outcomes in $B$.
  \part List all of the basic outcomes in $A \cap B$
  \part List all of the basic outcomes in $A \cup B$.
  \part If male and female births are equally likely, what is the probability of $A$?
\end{parts}

\question Let $B = A^c$. Are $A$ and $B$ mutually exclusive? Are they collectively exhaustive? Why?

\question State each of the three axioms of probability, aka the \emph{Kolmogorov Axioms}.

\question Suppose we carry out a random experiment that consists of flipping a fair coin twice.
	\begin{parts}
		\part List all the basic outcomes in the sample space.
		\begin{solution}
			$S = \{HH, HT, TT, TH\}$
		\end{solution}
		\part Let $A$ be the event that you get at least one head. List all the basic outcomes in $A$.
		\begin{solution}
			$A = \{HH, HT, TH\}$
		\end{solution}
		\part List all the basic outcomes in $A^c$. 
		\begin{solution}
			$A^c = \{TT\}$
		\end{solution}
		\part What is the probability of $A$? What is the probability of $A^c$?
		\begin{solution}
			$P(A) = 3/4 = 0.75$ and $P(A^c) = 1/4$
		\end{solution}
	\end{parts}

\question Calculate the following:
\begin{parts}
  \part $5!$
  \begin{solution}
    120
  \end{solution}
  \part $\displaystyle \frac{100!}{98!}$
  \begin{solution}
    9900
  \end{solution}
  \part $\displaystyle {5 \choose 3}$
  \begin{solution}
    10
  \end{solution}
\end{parts}


\question 
\begin{parts}
  \part How many different ways can we choose a President and Secretary from a group of 4 people if the two offices must be held by different people?
  \part How many different committees with two members can we form a group of 4 people, assuming that the order in which we choose people for the committee doesn't matter. 
\end{parts}

\question Suppose that I flip a fair coin 5 times.
\begin{parts}
  \part How many basic outcomes contain exactly two heads? 
  \part How many basic outcomes contain exactly three tails?
  \part How many basic outcomes contain exactly one heads?
  \part How many basic outcomes contain exactly four tails?
\end{parts}

\question Explain why $\displaystyle{n \choose r} = {n \choose n-r}$.

%\question Suppose I deal two cards at random from a well-shuffled deck of 52 playing cards. What is the probability that I get a pair of aces? 
%	\begin{solution}
%	You can either solve this assuming that order doesn't matter:
%		$$\frac{\binom{4}{2}}{\binom{52}{2}} = \frac{4!/(2!\times 2!)}{52!/(50!  \times 2!)} = \frac{6}{(52\times 51)/2}= 6/1326 = 1/221$$
%		or that it does:
%		$$\frac{P^4_2}{P^{52}_2} = \frac{4!/2!}{52!/50!} =\frac{(4\times 3)}{(52\times 51)} = 12/2652 = 1/221$$
%		In either case, the answer is the same: $1/221  \approx 0.005$
%	\end{solution}

  \question Suppose that I choose two distinct numbers at random from the set $\left\{ 1, 2, 3, 4, 5, 6, 7, 8, 9 \right\}$. What is the probability that both are odd?
  \begin{solution}
    This solution assumes that order doesn't matter.
    You could also assume that it does matter and get the same answer.
    There are $\displaystyle {9 \choose 2} = 36$ equally likely ways to choose 2 items from a set of 9. Of these, there are $\displaystyle {5 \choose 2} = 10$ ways to choose 2 of the 5 odd numbers.
    Hence the probability is $10/36 = 5/18$.
  \end{solution}

%%%%%%%%%%%%%%%%%%%%%%%%%%%%%%%%%%%%%%%%%%%%%%%%%%%%%%%%%%%%%%%%%%
\fullwidth{\section*{Lecture \#6 -- Basic Probability II}}

\question State and prove the \emph{complement rule}.

\question State the \emph{multiplication rule}, and compare it to the definition of conditional probability.

\question Mark each statement as TRUE or FALSE. If FALSE, give a one sentence explanation.
\begin{parts}
  \part If $A \subseteq B$ then $P(A) \geq P(B)$.
  \begin{solution}
    FALSE: this is the logical consequence rule with the inequality sign going in the \emph{wrong direction}. 
  \end{solution}
  \part For any events $A$ and $B$, $P(A\cap B) = P(A)P(B)$.
  \begin{solution}
    FALSE: this only holds if $A$ and $B$ are independent.
  \end{solution}
  \part For any events $A$ and $B$, $P(A\cup B) = P(A) + P(B) - P(A\cap B)$.
  \begin{solution}
    TRUE: this is the addition rule.
  \end{solution}
\end{parts}

\question Suppose that $P(B) = 0.4$, $P(A|B) = 0.1$ and $P(A|B^c) = 0.9$. 
\begin{parts}
  \part Calculate $P(A)$.
\begin{solution}
  By the law of total probability,
  \[
    P(A) = P(A|B)P(B) + P(A|B^c)P(B^c) = 0.1 \times 0.4 + 0.9 \times 0.6 = 0.58
  \]
\end{solution}
  \part Calculate $P(B|A)$.
  \begin{solution}
    By Bayes' rule,
    \[
      P(B|A) = \frac{P(A|B)P(B)}{P(A)} = \frac{0.1 \times 0.4}{0.58} = 2/29 \approx 0.07
    \]
  \end{solution}
\end{parts}

\question Define statistical independence. How is it related to conditional probability, and what does it mean intuitively?

\question State and prove the law of total probability for $k = 2$.

\question Find the probability of getting \emph{at least} one six if you roll a fair, six-sided die three times.
		\begin{solution}
			Using the complement rule:
				$$P(\mbox{At Least One Six}) = 1 - P(\mbox{No Sixes})$$
			And by independence:
				$$P(\mbox{No Sixes}) = 5/6 \times 5/6 \times 5/6 = 125/216$$
			Hence, 
			$$P(\mbox{At Least One Six}) = 1 - 125/216 = 91/216 \approx 0.42$$
		\end{solution}
	
%\question Suppose everyone in a class of one hundred students flips a fair coin five times.
%	\begin{parts}
%    \part What is the probability that a given student in the class gets five heads in a row? 
%			\begin{solution}
%				$(1/2)^5 = 1/32\approx 0.03$
%			\end{solution}
%	 	\part What is the probability that at least one student gets five heads in a row?
%	 	\begin{solution}
%	 	Use the complement rule: let $A$ be the event that at least one person gets five heads in a row. Calculate the probability that no one gets 5 heads in a row as follows:
%	 		$$P(A^c) = (1 - 1/2^5)^{100} = (31/32)^{100}\approx 0.04$$
%	 		Hence the desired probability is about $0.96$.
%	 	\end{solution}
%	\end{parts}

\question Suppose a couple decides to have three children. Assume that the sex of each child is independent, and the probability of a girl is $0.48$, the approximate figure in the US. 
	\begin{parts}
		\part How many basic outcomes are there for this experiment? Are they equally likely?
		\begin{solution}
			There are two possible outcomes for each birth, so by the multiplication rule for counting, the total number of possibilities is $2\times 2\times 2 = 8$.
			They are not equally likely because each child is more likely to be a boy than a girl. The outcome BBB is most likely, followed by outcomes with two boys, and then outcomes with one boy. The outcome GGG is least likely.
			\end{solution}
		\part What is the probability that the couple has \emph{at least one} girl?
			\begin{solution}
				Use the Complement Rule and independence to calculate the probability of no girls, i.e.\ all boys:
					$$0.52 \times 0.52 \times 0.52 \approx 0.14$$
				Hence, the probability of at least one girl is approximately $1 - 0.14 = 0.86$
			\end{solution}
	\end{parts}

  \question Let $A$ and $B$ be two arbitrary events. Use the addition rule and axioms of probability to establish the following results.
\begin{parts}
  \part Show that $P(A\cup B) \leq P(A) + P(B)$. (This is called \emph{Boole's Inequality}.)
  \begin{solution}
   By the Addition Rule $P(A\cup B) = P(A) + P(B) - P(A\cap B)$. The result follows since $P(A\cap B) \geq 0$ by the first axiom of probability. 
  \end{solution}
  \part Show that $P(A\cap B) \geq P(A) + P(B) - 1$. (This is called \emph{Bonferroni's Inequality})
  \begin{solution}
   Rearranging the Addition Rule, $P(A\cap B) = P(A) + P(B) - P(A\cup B)$. The result follows since $P(A\cup B)$ is at most one by the first axiom of probability.
  \end{solution}
\end{parts}

%\question Suppose I flip a fair coin and roll a single fair die at the same time. Define the events 
%\\ $A =$ the coin comes up tails 
%\\ $B =$ the die shows a 3 \emph{or} 5
%\\ $C =$ the die shows an \emph{odd} number 
% 	\begin{parts} 
%    \part Calculate $P(B|C)$.
%    \begin{solution}
%      \[P(B|C) = P(B\cap C)/P(C) =  (1/3)/(1/2) = 2/3\] 
%    \end{solution}
%    \part Calculate $P(A \cap B)$.
%    \begin{solution}
%      Since the dice roll and coin flip are independent, we have $P(A \cap B) = P(A)P(B) = (1/2) \times (1/3) = 1/6$. 
%    \end{solution}
%    \part Calculate $P(A \cup B)$.
%    \begin{solution}
%      $P(A \cup B) = P(A) + P(B) - P(A \cap B) = 1/2 + 1/3 -1/6 = 3/6 + 2/6 - 1/6 = 4/6 = 2/3$.  
%    \end{solution}
%
% 	\end{parts}

\question Let $A$ and $B$ be two mutually exclusive events such that $P(A)>0$ and $P(B)>0$. Are $A$ and $B$ independent? Explain why or why not.
\begin{solution} 
  They are not independent: knowing that one has occurred means that the other \emph{cannot have occurred}.
  You can also show this mathematically.
  Since $A$ and $B$ are mutually exclusive, $P(A\cap B) = 0$.
  But independence requires that $P(A\cap B) = P(A)P(B)$.
  Since neither $P(A)$ nor $P(B)$ is zero, it follows that the events cannot be independent.
\end{solution}

\question Molly the meteorologist determines that the probability of rain on Saturday is 50\%, and the probability of rain on Sunday is also 50\%.
Adam the anchorman sees Molly's forecast and summarizes it as follows:  ``According to Molly we're in for a wet weekend. There's a 100\% chance of rain this weekend: 50\% on Saturday and 50\% on Sunday.'' Is Adam correct? Why or why not? 
			\begin{solution}
				Adam is incorrect. 
        Let $A$ be the event that it rains on Saturday, $B$ be the event that it rains on Sunday, and $C$ be the event that it rains on the weekend.
        By the addition rule $P(C) = P(A) + P(B) - P(A\cup B)$, so Adam is only correct if $P(A\cup B) = 0$, in other words he is only correct if it is \emph{impossible} for it to rain on both Saturday and Sunday.
			\end{solution}


\question Suppose I throw two fair, six-sided dice once. Define the following events:
	\begin{eqnarray*}
		E &=& \mbox{The first die shows 5}\\
		F &=& \mbox{The sum of the two dice equals 7}\\
		G &=& \mbox{The sum of the two dice equals 10}
	\end{eqnarray*}
	\begin{parts}
		\part Calculate $P(F)$.
			\begin{solution}
				Of the 36 basic outcomes of the experiment, the pairs (1,6), (6,1), (2,5), (5,2), (3,4), and (4,3) sum to 7. Hence the probability is 1/6.
			\end{solution}
		\part Calculate $P(G)$.
			\begin{solution}
				Of the 36 basic outcomes of this experiment, the pairs (5,5), (4,6), and (6,4) sum to 10. Hence the probability is $3/36 = 1/12$.
			\end{solution}
		\part Calculate $P(F|E)$.
			\begin{solution}
			By the definition of conditional probability,
			 $$P(F|E) = \frac{P(F\cap E)}{P(E)}$$
			 We know that $P(E) = 1/6$. The only way that $F\cap E$ can occur is if we roll (5,7). Hence $P(F\cap E) = 1/36$. Thus, $P(F|E) = (1/36)/(1/6) = 6/36 = 1/6$.
			\end{solution}
		\part Calculate $P(G|E)$.
			\begin{solution}
				Again, by the definition of conditional probability,
					$$P(G|E) = \frac{P(G\cap E)}{P(E)}$$
				As before, $P(E) = 1/6$. The only way for $G\cap E$ to occur is if we roll (5,5). Hence $P(G|E) = (1/36)/(1/6) = 6/36 = 1/6$.
			\end{solution}
	\end{parts}


\fullwidth{\section*{Lecture \#7 -- Basic Probability III / Discrete RVs I}}

\question What is the base rate fallacy? Give an example.

\question Derive Bayes' Rule from the definition of conditional probability.

\question What are two names for the \emph{unconditional} probability in the numerator of Bayes' rule?

\question When is it true that $P(A|B) = P(B|A)$? Explain.

\question Of women who undergo regular mammograms, two percent have breast cancer. If a woman has breast cancer, there is a 90\% chance that her mammogram will come back positive. If she does \emph{not} have breast cancer there is a 10\% chance that her mammogram will come back positive. Given that a woman's mammogram has come back positive, what is the probability that she has breast cancer? 
	\begin{solution}
		 Let $B$ be the event that a given woman has breast cancer and $M$ be the event that her mammogram comes back positive. By Bayes' Rule,
		 		\[
          P(B|M) = \frac{P(M|B)P(B)}{P(M)}
        \]
	By the law of total probability, 
		 			\begin{align*}
		 			P(M) &= P(M|B)P(B) + P(M|B^c)P(B^c)\\
		 				&= 0.9 \times 0.02 + 0.1 \times 0.98  = 0.018 + 0.098 = 0.116
		 			\end{align*}
		 	Hence,
		 		\[
          P(B|M) = \frac{0.9 \times 0.02}{0.116} = \frac{0.018}{0.116} \approx 0.16
        \]
	\end{solution}

%  \question Sherlock Holmes has gone away on vacation, instructing Dr.\ Watson to water the flowers in his absence. 
%  Unfortunately Watson has a rather poor memory: the probability that he will remember to water the flowers is only 2/3.
%  The flowers weren't in the best shape when Holmes left: even if watered the probability that they will wither and die before Holmes returns is 1/2.
%  If they aren't watered, the probability that they will wither and die increases to 3/4. 
%  Holmes returns to find that his flowers have died.
%  What is the probability that Watson forgot to water them?
%  \begin{solution}
%    By Bayes' Rule:
%    \begin{equation*}
%      P(\mbox{Forget}|\mbox{Die}) = \frac{P(\mbox{Die}|\mbox{Forget})P(\mbox{Forget})}{P(\mbox{Die})}
%    \end{equation*}
%    We calculate the denominator using the law of total probability as follows:
%    \begin{eqnarray*}
%      P(\mbox{Die}) &=& P(\mbox{Die}|\mbox{Forget})P(\mbox{Forget})+ P(\mbox{Die}|\mbox{Remember})P(\mbox{Remember})\\
%      &=& 3/4 \times 1/3 + 1/2 \times 2/3 = 3/12 + 2/6 = 7/12
%    \end{eqnarray*}
%    Thus $P(\mbox{Forget}|\mbox{Die}) = (3/12)/(7/12) = 3/7$.
%    It is more likely than not that Watson forgot to water the flowers.
%  \end{solution}

%\question I have two six-sided dice in my pocket: one fair die and one loaded die. The fair die has the usual probabilities, but the probability of getting a 6 when rolling the loaded die is 1/2. Suppose I reach into my pocket and draw one of the two dice at random (both are equally likely to be drawn). I roll this randomly chosen die and get a 6. What is the probability that I drew the loaded die? 
%	\begin{solution}
%		Let $L$ be the event that I draw the loaded die, $F$ be the event that I draw the fair die and $6$ be the event that I roll a six. By Bayes' rule, we have
%			$$P(L|6) = \frac{P(6|L)P(L)}{P(6)}$$
%Calculating the denominator by the Law of Total Probability, we have
%		\begin{eqnarray*}			
%			P(6) &=& P(6|L)P(L) + P(6|F)P(F) \\
%			&=& 1/2 \times 1/2 + 1/6\times 1/2\\
%			&=&	1/4 + 1/12 \\
%			&=& 1/3		
%			\end{eqnarray*}
%		Hence,
%			$$P(L|6) = \frac{1/4}{1/3} = 3/4 = 0.75$$
%	\end{solution}
	
	

\question The Triangle is a neighborhood that once housed a chemical plant but has become a residential area. Two percent of the children in the city live in the Triangle, and fourteen percent of these children test positive for excessive presence of toxic metals in the tissue. For children in the city who do not live in the Triangle, the rate of positive tests is only one percent. If we randomly select a child who lives in the city and she tests positive, what is the probability that she lives in the Triangle?
		\begin{solution}
		By the law of total probability
		\begin{align*}
		P(M) &= P(M|T)P(T) + P(M|T^c)P(T^c)\\
			&= 0.14 \times 0.02 +  0.01 \times 0.98\\
			&= 0.0028 + 0.0098 \\
			&= 0.0126
		\end{align*}
		and by Bayes' Rule:
				\begin{align*}
				P(T|M) &= \frac{P(M|T)P(T)}{P(M)}\\
					&= \frac{0.0028}{0.0126} = 2/9 \approx 0.22
				\end{align*}
			\end{solution}


\question Three percent of \emph{Tropicana} brand oranges are already rotten when they arrive at the supermarket. In contrast, six percent of \emph{Sunkist} brand oranges arrive rotten. A local supermarket buys forty percent of its oranges from \emph{Tropicana} and the rest from \emph{Sunkist}. 
		Suppose we randomly choose an orange from the supermarket and see that it is rotten. What is the probability that it is a \emph{Tropicana}?
		\begin{solution}
			By the law of total probability:
				\begin{align*}
				P(R) &= P(R|T)P(T) + P(R|T^c)P(T^c)\\
					&= 0.03 \times 0.4 + 0.06 \times 0.6\\
					&= 0.012 + 0.036 \\
					&= 0.048
				\end{align*}
		 and by Bayes' Rule:
			\begin{align*}
			P(T|R) &= \frac{P(R|T)P(T)}{P(R)}\\
					&= \frac{0.012}{0.048} = 1/4 = 0.25
			\end{align*}
		\end{solution}

\question Define the terms \emph{random variable}, \emph{realization}, and \emph{support set}.

\question What is the probability that a RV takes on a value outside of its support set?

\question What is the difference between a \emph{discrete} and \emph{continuous} RV? 

\question What is a \emph{probability mass function}? What two key properties does it satisfy?

%\question The \emph{Rademacher} RV is equally likely to take any value in the set $\left\{ -1,1 \right\}$ and never takes on any value outside this set. Write out and sketch its probability mass function.
%
\fullwidth{\section*{Lecture \#8 -- Discrete RVs II}}

\question Define the term \emph{cumulative distribution function} (CDF).
How is the CDF of a discrete RV $X$ related to its pmf?

\question Let $X$ be a RV with support set $\left\{-1, 1 \right\}$ and $p(-1) = 1/3$.
Write down the CDF of $X$. 
\begin{solution}
$F(x_0) = \left\{\begin{array}{l} 0,\, x_0 < -1 \\ 1/3, \, -1 \leq x_0 < 1\\ 1,\, x_0 \geq 1\end{array} \right.$
\end{solution}

\question Write out the support set, pmf, and CDF of a Bernoulli$(p)$ RV.

\question Define the term \emph{parameter} as it relates to a random variable. Are parameters constant or random?

\question Let $X$ be a RV with support set $\left\{0, 1, 2 \right\}$, $p(1) = 0.3$, and $p(2) = 0.5$. Calculate $E[X]$.
\begin{solution}
  $E(X) = 0 \times 0.2 + 1 \times 0.3 + 2 \times 0.5 = 1.3$
\end{solution}


\emph Let $X$ be a discrete RV. Define the expected value $E[X]$ of $X$. Is $E[X]$ constant or random? Why?

\question Suppose $X$ is a RV with support $\{-1, 0, 1\}$ where $p(-1)=q$ and $p(1) = p$. What relationship must hold between $p$ and $q$ to ensure that $E[X] = 0$?
		\begin{solution}
				By the complement rule $p(0) = 1 - p - q$.
        Hence, 
        \[
          E[X] = -1 \cdot q + 0 \cdot (1-p-q) + p\cdot 1 = p-q
        \]
        so that $E[X] = 0$ if and only if $p = q$.
    \end{solution}

\question Let $X$ be a discrete RV and $a, b$ be constants. 
Prove that $E[a + bX] = a + bE[X]$.

\question Suppose that $E[X]=8$ and $Y= 3 + X/2$. Calculate $E[Y]$.
\begin{solution}
$E(Y) = 3 + E(X)/2 = 7$
\end{solution}

\question Suppose that $X$ is a discrete RV and $g$ is a function. Explain how to calculate $E[g(X)]$. Is this the same thing as $g\left(E[X]\right)$?

\question Let $X$ be a RV with support set $\left\{ -1, 1 \right\}$ and $p(-1) = 1/3$. Calculate $E[X^2]$.
\begin{solution}
  $E[X^2] = (-1)^2 \times 1/3 + (1)^2 \times 2/3 = 1$
\end{solution}

\question Let $X$ be a RV with support set $\left\{ 2,4 \right\}$, $p(2) = 1/2$ and $p(4) = 1/2$. Mark each of the following claims as TRUE or FALSE, either by appealing to a result from class, or by directly calculating both sides of the equality.
\begin{parts}
  \part $E[X+10] = E[X] + 10$
  \begin{solution}
    TRUE by the linearity of expectation.
  \end{solution}
  \part $E[X/10] = E[X]/10$
  \begin{solution}
    TRUE by the linearity of expectation.
  \end{solution}
  \part $E[10/X] = 10/E[X]$
  \begin{solution}
    FALSE. By direct calculation:
    \begin{align*}
      E[10/X] &= 1/2 \times 10/2 + 1/2 \times 10/4 = 15/4 = 3.75\\
      10/E[X] &= 10/(1/2 \times 2 + 1/2 \times 4) = 10/3
    \end{align*}
  \end{solution}
  \part $E[X^2] = \left(E[X]\right)^2$
  \begin{solution}
   FALSE. By direct calculation: 
    \begin{align*}
      E[X^2] &= 1/2 \times 2^2 + 1/2 \times 4^2 = 10\\
      \left( E[X] \right)^2 &= \left( 1/2 \times 2 + 1/2 \times 4 \right)^2 = 9
    \end{align*}
  \end{solution}
  \part $E[5X + 2]/10 = \left(5E[X] + 2\right)/10$
  \begin{solution}
    TRUE by the linearity of expectation.
  \end{solution}
\end{parts}


\fullwidth{\section*{Lecture \#9 -- Discrete RVs III}}

\question Define the \emph{variance} and \emph{standard deviation} of a RV $X$. Are these constant or random?

\question Explain how to use our formula for $E[g(X)]$ to calculate the variance of a discrete RV. 

\question Write down the shortcut formula for variance, and use it to calculate $Var(X)$ where $X\sim\mbox{Bernoulli}(p)$. 

\question Let $X$ be a random variable and $a,b$ be constants. 
Prove that $Var(a + bX) = b^2 Var(X)$.

\question Define the Bernoulli$(n,p)$ RV in terms of independent Bernoulli trials, and write down its support set and probability mass function.

\question Substitute $n=1$ into the pmf of a Binomial$(n,p)$ RV and show that you obtain the pmf of a Bernoulli$(p)$ RV.
	\begin{solution}
		The pmf for a Binomial$(n,p)$ RV is
		$$p(x) = {n \choose x} p^x (1-p)^{n-x}$$
		with support $\{0, 1, 2\hdots, n\}$. Setting $n=1$ gives,
		$$p(x) = p(x) = {1 \choose x} p^x (1-p)^{1-x}$$
		with support $\{0,1\}$. Plugging in each realization in the support, and recalling that $0! = 1$, we have
			$$p(0) = \frac{1!}{0!(1-0)!} p^0 (1-p)^{1-0} = 1 - p$$
		and
		$$p(1) = \frac{1!}{1!(1-1)!} p^1 (1-p)^0 = p$$
		which is exactly how we defined the Bernoulli Random Variable.
	\end{solution}

\question A multiple choice quiz has 12 questions, each of which has 5 choices. To pass you need to get at least 8 of them correct. Nina forgot to study, so she simply guesses at random.
\begin{parts}
  \item Let the random variable $X$ denote the number of questions that Nina gets correct on the quiz. What kind of random variable is $X$? Specify all parameter values.
  \item Calculate the probability that Nina passes the quiz.
\end{parts}

		
\question Define the \emph{joint probability mass function} $p_{XY}$ of two discrete RVs $X$ and $Y$ and list its two key properties.

\question What is the difference between a joint pmf and a marginal pmf?

\question Can you calculate a marginal pmf from a joint? How? Can you calculate a joint pmf from a marginal pmf?

\question Suppose that $X$ is a random variable with support $\{1,2\}$ and $Y$ is a random variable with support $\{0,1\}$ where $X$ and $Y$ have the following joint pmf: 
			\begin{eqnarray*}
				p_{XY}(1,0) = 0.20, && p_{XY}(1,1) = 0.30 \\
				p_{XY}(2,0) = 0.25, && p_{XY}(2,1) = 0.25
			\end{eqnarray*}
	\begin{parts}
  \item Express the joint pmf in a table with $X$ in the \emph{rows}, as we did in class. 
			\begin{solution}
			\begin{center}
\begin{tabular}{|cc|cc|}
\hline
&&\multicolumn{2}{c|}{$X$}\\
&&1 & 2\\
\hline
\multirow{2}{*}{$Y$}
&0& \multicolumn{1}{|c}{0.20} & 0.25\\
&1& \multicolumn{1}{|c}{0.30} & 0.25\\
\hline
\end{tabular}
\end{center}
			\end{solution}
		\item Using the table, calculate the marginal pmfs of $X$ and $Y$.
			\begin{solution}
				\begin{eqnarray*}
					p_X(1) &=&p_{XY}(1,0) + p_{XY}(1,1)=0.20+0.30 = 0.50 \\
					p_X(2) &=&p_{XY}(2,0) + p_{XY}(2,1)=0.25 + 0.25 = 0.50 \\
					p_Y(0) &=&p_{XY}(1,0) + p_{XY}(2,0) = 0.20 + 0.25 = 0.45 \\
					p_Y(1) &=& p_{XY}(1,1) + p_{XY}(2,1) = 0.30 + 0.25 = 0.55
				\end{eqnarray*}
			\end{solution}
\end{parts}



\end{questions}


\end{document}
