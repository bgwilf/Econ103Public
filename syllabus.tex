\documentclass[11pt, letterpaper]{article}
\usepackage{geometry}
\geometry{margin=1in} 
\usepackage{setspace}
\linespread{1}
\usepackage{hyperref}
\usepackage{totcount}
\usepackage{termcal}
\usepackage{enumerate}
\usepackage{fancybox}
\usepackage{amsmath, amssymb}

%%%%%%%%%%%%% A Command to create an automatically numbered Quiz Icon in the course calendar.
\newtotcounter{quizcounter}
\newcommand{\quiz}{\addtocounter{quizcounter}{1}\fbox{\textbf{Quiz \#\arabic{quizcounter}}}}

%%%%%%%%%%%%% A Command to create an automatically problem set icon in the course calendar.
\newtotcounter{pscounter}
\newcommand{\ps}{\addtocounter{pscounter}{1}\fbox{\textbf{PS \#\arabic{pscounter} Due}}}


% Some useful commands (our classes always meet either on Monday and Wednesday 
% or on Tuesday and Thursday)
\newcommand{\MWClass}{%
\calday[Monday]{\classday} % Monday
\skipday % Tuesday (no class)
\calday[Wednesday]{\classday} % Wednesday
\skipday % Thursday (no class)
\skipday % Friday 
\skipday\skipday % weekend (no class)
}

\newcommand{\TRClass}{%
\skipday % Monday (no class)
\calday[Tuesday]{\classday} % Tuesday
\skipday % Wednesday (no class)
\calday[Thursday]{\classday} % Thursday
\skipday % Friday 
\skipday\skipday % weekend (no class)
}

\newcommand{\Holiday}[2]{%
\options{#1}{\noclassday}
\caltext{#1}{#2}
}

\begin{document}


\thispagestyle{plain}

\begin{center}
\Large
\sc
Statistics for Economists\\
\large
Economics 103\\
\large
Spring 2019
\end{center}



\normalsize

\noindent \textbf{Course Instructor:} Francis DiTraglia \\
Office: PCPSE 630\\
Office Hours: M 3--4pm, R 4--5pm

\medskip


\noindent \textbf{Recitation Instructors:}

\medskip
\noindent

\begin{tabular}{llll}
  & Assa Cohen & Philippe Goulet Coulombe & Gabrielle Vasey\\
Office:& PCPSE 208& PCPSE 208 & PCPSE 500 \\ 
Office Hours:& R noon--2pm & M 2--3pm & M 9--11am   
\end{tabular}

\medskip
 
\noindent \textbf{Lecture Time and Location:} TR 9-10:30AM ANNS 110 

\medskip

\noindent \textbf{Recitations Times and Locations:}
\medskip \noindent

\begin{tabular}{llllll}
	Section 201 & Section 202 & Section 203 & Section 204 & Section 205 & Section 206\\ 
  Philippe & Assa & Gabrielle & Assa & Gabrielle & Philippe \\
	MCNB 285 & WILL 204 & MCNB 167-8 & WILL 204 & MCNB 285 & WILL 216 \\ 
  M 10--11am & F 9--10am & M noon--1pm & F 10--11am & M 11am--noon & M 1--2pm
\end{tabular}


\medskip

\medskip
 
\noindent \textbf{Course Website:} \url{http://ditraglia.com/Econ103Public} At this url you can download all lecture slides, homework problems, etc.
You can view your grades and log-on to the course discussion forum, Piazza, at \url{https://canvas.upenn.edu}.

\medskip

\noindent \textbf{Email Policy:}
Please direct all written communication concerning Econ 103 to the course discussion forum -- Piazza -- rather than to the instructors' personal email accounts.
For personal issues, use Piazza's private messaging feature to communicate directly with the course instructors. 

\medskip



\noindent \textbf{Course Description:} 
This course will teach you how to learn from data and understand uncertainty using the ideas of probability theory and statistics. 
After completing this course you will be able to carry out simple statistical analyses of your own using the computer package R.


\medskip


\noindent \textbf{Prerequisites:} 
The prerequisites for this course are Math 104 followed by 114 or 115. 



\medskip

\noindent \textbf{Textbook:} 
\emph{Introductory Statistics for Business and Economics}, 4\textsuperscript{th} Edition by Thomas H.\ and Ronald J.\ Wonnacott (WW4). 
Used copies are available on \href{http://tinyurl.com/ECON103-2013A}{Amazon} for under \$20.
I will list suggested readings on Piazza every week. 
While I encourage you to complete the reading assignments, my lecture slides and homework assignments are the final authority on course material.

\medskip


\noindent \textbf{Required Software:} 
We will use the statistical package R via a front-end called RStudio throughout the course. 
Both R and RStudio are free and open source. 
I will post installation instructions on Piazza.
%Installation instructions appear on the last page of this syllabus.
RStudio is also available in the Undergraduate Data Analysis Lab (UDAL) in McNeil rooms 104 and 108--9. 
You will learn to use R in lecture, recitations, and through as series of tutorials that I will assign as homework. (See ``R Tutorials'' below.)  
%Additional R resources are listed on the last page of this syllabus.
As part of the course, every student will also receive a free premium subscription to \href{https://datacamp.com}{DataCamp}.

\medskip

\noindent \textbf{Recommended Texts:} 
I recommend two texts for students who need extra help with the course material. 
First is the \emph{Student Workbook to accompany Introductory Statistics for Business and Economics 4\textsuperscript{th} Edition}. 
Used copies are available on \href{http://www.amazon.com/gp/offer-listing/0471508993/sr=/qid=/ref=olp_page_2?ie=UTF8&colid=&coliid=&condition=all&me=&qid=&shipPromoFilter=0&sort=sip&sr=&startIndex=10}{Amazon}. 
The workbook contains fully worked out solutions to all odd-numbered problems from the textbook along with additional practice problems and solutions.
If you're having trouble with R and prefer a printed book to the free online resources listed below, I suggest consulting \emph{The R Student Companion} by Brian Dennis.


\newpage

\noindent \textbf{Departmental Course Policies: } 
All Economics Department course policies are in force in Econ 103 even if not explicitly listed on this syllabus. 
See: \url{https://economics.sas.upenn.edu/undergraduate/course-information/course-policies} for full details.

\medskip


\noindent \textbf{Academic Integrity: } 
All suspected violations of the code of academic integrity as set forth in the Pennbook will be reported to the Office of Student Conduct. 
Confirmed violations will result in a failing grade for the course. 
We will check identification cards at exams so please to bring yours.

\medskip

\noindent \textbf{Piazza:} 
We will use an online discussion forum called Piazza, accessible via \href{http://upenn.instructure.com}{Canvas}, for all written communication in this course.
We will use Piazza to make course announcements, answer questions about course material and respond to private messages from individual students regarding personal issues.
%By asking your question and getting an answer on Piazza, you create a positive externality: other students benefit from your questions and you benefit from theirs.
%You can even post anonymously if asking questions publicly makes you uncomfortable.
%The instructor and RIs will actively moderate Piazza both to answer questions and approve (or correct) answers written by your fellow-students.
We will award extra credit for constructive questions, answers, and notes that you post on Piazza, even if you post anonymously.
See below under ``Extra Credit'' for details.
%As mentioned above under ``Email Policy,'' please direct all written communication for Econ 103 to Piazza rather than the instructors' personal email accounts.

\section*{Homework}
Homework will consist of three components: review questions, extension problems, and online R tutorials. 
With the exception of R tutorials, homework will neither be collected nor graded.

\medskip
\noindent \textbf{Review Questions:} 
Corresponding to each lecture of the semester are 10--15 ``review questions'' covering the basic knowledge that you will need to acquire this semester to pass the course.
These are available from the \href{http://ditraglia.com/Econ103Public}{Course Website}.
After a given lecture, and before the next one, you should solve all of the associated review questions. 
To give you an incentive to keep up with the course material, all quiz questions for the course will be randomly selected from this list: see ``Quizzes'' below.
Review questions are straightforward.
Most are taken directly from the lecture slides, and nearly all of the rest involve straightforward calculations similar if not identical to those from the lecture.
As such we will not circulate solutions to review questions.
Compiling your own solutions is an important part of studying for the course.
We will be happy to discuss any of the review questions with you in person or on Piazza, and you are most welcome to discuss them with your classmates.
Be warned, however, that memorizing answers written by a classmate is a risky strategy.
It may get you through the quiz, but will leave you unprepared for exams.

\medskip
\noindent \textbf{Extension Problems:} 
Unlike review questions, extension problems are designed to give you a deeper understanding of the lecture material and challenge you to apply what you have learned in new settings.
Extension problems should only be attempted \emph{after} you have completed the corresponding review problems.
%Extension problems are intended to be completed \emph{after} you the quiz for a given lecture. 
%For example, Quiz \#2 on January 31st covers lectures 3--4. 
%This means that you should complete the extension problems for lectures 3--4 over the weekend of February 1st--3rd.
As an extra incentive to keep up with the course material, each exam of the semester will contain at least one problem taken \emph{verbatim} from the extension problems.
We will circulate solutions to the relevant extension questions the weekend before each exam.
Like the review questions, extension problems are available from the \href{http://ditraglia.com/Econ103Public}{Course Website}.

\medskip
\noindent \textbf{R Tutorials:}
While I will provide numerous R code examples in lecture, you cannot really learn to program by reading a set of slides: you have to spend time working with R yourself.
As such, I will assign a number of online R tutorials from \href{https://datacamp.com}{DataCamp} as homework to complement the in-person R instruction that you will receive in lecture and recitations.
Although these will not be graded, completion of these assignments will count towards your recitation grade: see ``Recitations'' below.
Later in the semester, R material will appear in extension problems and midterm and final exams.




\section*{Assignments and Grading}

Grades for this course will be determined based on recitations, quizzes, two in-class midterms, and a comprehensive final examination that will take place during the exam period:
	\begin{equation*}
	\begin{split}
    \mbox{Grade} = 10\% \times \mbox{Recitation} + 20\% \times \mbox{Quizzes} + 20\% \times \mbox{Midterm1} + 20\% \times \mbox{Midterm2} + 30\% \times \mbox{Final}.
	\end{split}
	\end{equation*}
You can earn extra credit worth up to 5\% of your course grade: see ``Extra Credit'' for details.

\medskip 

\noindent \textbf{Course Curve:}
There will be no curve in Econ 103.
Grade boundaries will be as follows:
\begin{center}
\begin{tabular}{lr|lr|lr|lr}
  A+ & 98--100\%& B+ & 88--89\% & C+ & 78--79\% & D+ & 68--69\% \\
  A  & 92--97\% & B  & 82--87\% & C  & 72--77\% & D & 60--67\%\\
  A- & 90--91\% & B- & 80--81\% & C- & 70--71\% & F & 0--59\%\\
\end{tabular}
\end{center}
%If necessary, I will curve overall course scores (\emph{not} individual assignments) so that approximately 30\% fall in the A-range, 40-50\% fall in the B-range, and the bulk of the remaining 20-30\% fall in the C-range. 
%I reserve grades below a C-minus for those cases in which a student fails to attain a minimum level of basic competence in statistics, an absolute rather than relative standard. 
%If you are in danger of failing to meet this minimum standard, you will receive a course problem notice.
%I will only curve the course in your favor, so the most stringent possible grade boundaries are: A-range = 90-100, B-range = 80-89, C-range = 70-79, D-range = 60-69.
%(In this case, the top two points of each range would be a ``plus'' and the bottom two points a ``minus.'')

\medskip


\noindent \textbf{Quizzes:} 
There will be 11 in-class quizzes over the course of the semester: see the semester calendar on the last page of this document.
Each quiz will cover the two most recent lectures.
For example, Quiz \#1 will cover lectures 1--2 and Quiz \#5 will cover lectures 10--11.
When calculating your quiz average, I will drop your two lowest scores and weight the remaining quizzes evenly. 
There will be no makeup quizzes so be sure to use your two ``free skips'' carefully.
%Quizzes will not be returned and answers will not be posted but, your RI will be happy to go over your quiz with you.

\medskip

\noindent \textbf{Exams:} 
There will be two 70-minute in-class midterm exams and a 2-hour final exam during the exam period.
Dates appear on the last page of this document. 
Each midterm is worth 20\% and the final is worth 30\% of your grade.
All exams are comprehensive, but will focus on more recent material.
%For example, the final will focus on the last third of the course.
To give you a sense of the style and level of difficulty to expect, I have posted all of my past exams with full solutions on the \href{http://ditraglia.com/Econ103Public}{course website}.
There will be no makeup midterms: if you miss one midterm, your final exam will be worth 50\% to compensate; if you miss two midterms, it will be worth 70\%.
The makeup final will take place at the beginning of the Fall semester of 2019 and is outside of the instructor's control: the time and date are determined by the department, and eligibility is determined by the undergraduate chair.
%Please be aware that travel plans at the end of the semester, job internships etc.\ are \emph{not} a valid reason for missing an exam.
Note that registering for a course means that you certify that you will be present for the exam, unless one a small number of explicitly stated exceptions arises.
These do \emph{not} include travel plans or job internships: see ``Departmental Course Policies'' above.
Cheat sheets are not permitted on exams.
Scientific calculators are allowed but graphing calculators are not. 
You may write in pencil or pen on your exam as it will be photocopied before being returned to you.
We will check ID cards at each exam, so make sure to bring yours.

\medskip
 
\noindent \textbf{Recitations:} Attendance at your assigned recitation section is mandatory. Your grade for this component of the course will be determined by your recitation instructor (RI) and will be based on attendance, participation, and successful completion of any R tutorials assigned prior to recitation. (See ``R Tutorials.'') Consult your RI for the specific policy for missed recitations.

\medskip

\noindent \textbf{Regrade Requests:}
Exam regrade requests must be made in writing within a week of receiving your graded exam. 
As we re-grade the entire exam, your score could rise or fall. 
You may not discuss your answers with an RI or the instructor before submitting a regrade request. 


\medskip

\noindent \textbf{Extra Credit:} 
You can earn extra credit worth up to 5\% of your course grade for active participation on the course discussion board, Piazza.
Extra credit will be added to your overall course score after averaging all course assignments.
For example, if your overall course grade is 83\%, your final score after factoring in extra credit will be between 83\% and 88\%.
Extra credit will be awarded for constructive questions, answers, and notes on Piazza.
Even if you post anonymously, we can still award you extra credit for online participation.
Extra credit is discretionary and will calculated at the end of the semester.
As such, there is no precise formula that we can provide you in advance.
%Extra credit is intended as a modest reward for students who are taking the course seriously, regardless of whether they struggle or excel on quizzes and exams. 
%Attempts to game the system are not worth the effort: use your time to study instead.

%\section*{Installing R and RStudio} 
%N.B.\ these instructions will only work if you complete them in the \emph{correct order}.
%First, download and install R from \url{http://cran.r-project.org/}. 
%Next download and install RStudio by visiting \url{http://rstudio.org/download/desktop} and clicking the link listed under ``Recommended for Your System.'' 
%If you have trouble, post on Piazza or ask your RI or the instructor for help in office hours.
%As mentioned above, you will receive a free premium subscription to \href{http://datacamp.com}{DataCamp} as part of the course.
%DataCamp contains dozens of interactive online R tutorials, some of which will be assigned as homework for the course.
%Here are links to some additional free resources to help you learn R:
%\begin{itemize}
%		       \item \url{http://cran.r-project.org/other-docs.html}
%\item \url{http://www.twotorials.com/}
%			\item \url{http://www.r-bloggers.com/google-developers-r-programming-video-lectures/}
%		 	\item \url{http://cran.r-project.org/doc/contrib/Farnsworth-EconometricsInR.pdf}
% 			\item \url{http://www.ats.ucla.edu/stat/R/}
%\end{itemize}


\newpage


%NOTE: don't use leading zeros in dates! In other words, use 1/1/2014 rather than 01/01/2014

\begin{center}
\small
\begin{calendar}{1/14/2019}{16} %Date of Monday in first week of classes, NOT the date of the first class!
\setlength{\calboxdepth}{.25in}
\TRClass
% schedule
\caltexton{1}{Introduction \hfill \fbox{\emph{No Recitations 1/18 or 1/21}}}
\caltextnext{Summary Statistics I} 
\caltextnext{Summary Statistics II \hfill\quiz} 
\caltextnext{Regression I}
\caltextnext{Basic Probability I \hfill\quiz}
\caltextnext{Basic Probability II} 
\caltextnext{Basic Probability III / Discrete RVs I \hfill\quiz}
\caltextnext{Discrete RVs II}
%\caltextnext{\emph{Reserve Lecture} \hfill\quiz}
\caltextnext{Discrete RVs III \hfill\quiz} 
\caltextnext{Discrete RVs IV} 
\caltextnext{Continuous RVs I} 
\caltextnext{Continuous RVs II \hfill\quiz} 
\caltextnext{Sampling Dists.\ \& Estimation I}
\caltextnext{Sampling Dists.\ \& Estimation II \hfill\quiz} 
\caltextnext{Confidence Intervals I}
\caltextnext{Confidence Intervals II \hfill\quiz} 
\caltextnext{Confidence Intervals III} 
%\caltextnext{Confidence Intervals IV \hfill\quiz}
\caltextnext{Hypothesis Testing I \hfill\quiz}
\caltextnext{Hypothesis Testing II} 
\caltextnext{Hypothesis Testing III} 
\caltextnext{Hypothesis Testing IV\hfill\quiz} 
\caltextnext{Statistical Power}
\caltextnext{Regression II \hfill\quiz} 
\caltextnext{Regression III} 
\caltextnext{Reserve Lecture \hfill\quiz}
\caltextnext{Reserve Lecture} 
% ... and so on

% Holidays
\Holiday{1/15/2019}{\textbf{Winter Break -- No Lecture}}
%\Holiday{1/18/2018}{\textbf{R/RStudio Clinic instead of Lecture}} 
\Holiday{3/5/2019}{\textbf{Spring Break -- No Lecture}}
\Holiday{3/7/2019}{\textbf{Spring Break -- No Lecture}}

\options{2/19/2019}{\noclassday} % first midterm exam
\caltext{2/19/2019}{\shadowbox{\textbf{Midterm I} -- Material through Feb.\ 14th}}
\options{4/2/2019}{\noclassday} %second midterm exam
\caltext{4/2/2019}{\shadowbox{\textbf{Midterm II} -- Material through March 28th}}
%\options{4/24/2019}{\noclassday} % finals
%\caltext{4/24/2019}{\textbf{Final Review Session instead of Lecture}}
\options{5/2/2019}{\noclassday} % finals
\caltext{5/2/2019}{\textbf{Reading Day -- No Lecture}}
\end{calendar}
\end{center}


\noindent \textbf{Final Exam:} At the time of this writing (\today) the final exam for Econ 103 is scheduled for \href{https://www.registrar.upenn.edu/pdf_main/19A_Final_Exam_Tentative_01102019.pdf}{Tuesday, May 14\textsuperscript{th} at noon}, but this is subject to change.
Final confirmation of final times, along with room assignments, will be posted on the \href{https://www.registrar.upenn.edu/finals/index.html}{Registrar's Website} in April.

\end{document}
