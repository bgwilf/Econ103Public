\section{Candy Weighing Experiment}
%%%%%%%%%%%%%%%%%%%%%%%%%%%%%%%%%%%%%%%%
\begin{frame}
\frametitle{Weighing a Random Sample}
\begin{block}{Bag Contains 100 Candies}
Estimate total weight of candies by weighing a random sample of size 5 and multiplying the result by 20.
\end{block}
\begin{block}{Your Chance to Win}
The bag of candies and a digital scale will make their way around the room \alert{during the lecture}. Each student gets a chance to draw 5 candies and weigh them.
\end{block}
\begin{alertblock}{Student with closest estimate wins the bag of candy!}
\end{alertblock}

\end{frame}
%%%%%%%%%%%%%%%%%%%%%%%%%%%%%%%%%%%%%%%%
\begin{frame}
\frametitle{Weighing a Random Sample}
\begin{block}{Procedure}
When the bag and scale reach you, do the following:
\end{block}
\begin{enumerate}
\item Fold the top of the bag over and shake to randomize.
\item Randomly draw 5 candies \alert{without replacement}.
\item Weigh your sample and record the result \alert{in grams} along with your name on the sign-up sheet.
\item Replace your sample and shake again to re-randomize.
\item Pass bag and scale to next person.
\end{enumerate}
\end{frame}


%%%%%%%%%%%%%%%%%%%%%%%%%%%%%%%%%%%%%%%%
\section{Random Sampling Redux}
%%%%%%%%%%%%%%%%%%%%%%%%%%%%%%%%%%%%%%%%
\begin{frame}
  \frametitle{Sampling and Estimation}
  \begin{block}{Questions to Answer}
  \begin{enumerate}
    \item How accurately do sample statistics estimate population parameters?
    \item How can we quantify the uncertainty in our estimates?
    \item What's so good about random sampling?
  \end{enumerate}
  \end{block}
\end{frame}
%%%%%%%%%%%%%%%%%%%%%%%%%%%%%%%%%%%%%%%%

\begin{frame}
\frametitle{Random Sample}

\begin{block}{Verbal Definition from Lecture \#1}
Each member of population is chosen strictly by chance, so that: (1) selection of one individual doesn't influence selection of any other, (2) each individual is just as likely to be chosen, (3) every possible sample of size $n$ has the same chance of selection.
\end{block}

\begin{alertblock}{Mathematical Definition}
	$X_1, X_2, \hdots, X_n \sim \mbox{iid } f(x)$ if continuous\\
  $X_1, X_2, \hdots, X_n \sim \mbox{iid } p(x)$ if discrete
\end{alertblock}

\end{frame}

%%%%%%%%%%%%%%%%%%%%%%%%%%%%%%%%%%%%%%%%
\begin{frame}
  \frametitle{Random Sample Means \emph{Sample With Replacement}}


\begin{itemize}
  \item Sampling \emph{without replacement} creates dependence between samples (Extension Problem \#11).
  \item But if the population is large relative to the sample, this dependence is negligible: candy experiment isn't bogus!
\end{itemize}
\end{frame}

%%%%%%%%%%%%%%%%%%%%%%%%%%%%%%%%%%%%%%%%

\begin{frame}
\frametitle{Example: Sampling from Econ 103 Class List}
\begin{center}
\includegraphics[scale = 0.4]{./images/height_hist}
\end{center}
\alert{Use R to illustrate the in an example where we \emph{know} the population.  
Can't do this in the real applications, but simulate it on the computer\dots}
\end{frame}
%%%%%%%%%%%%%%%%%%%%%%%%%%%%%%%%%%%%%%%%

\begin{frame}
\frametitle{Sampling Distribution of $\bar{X}_n = \frac{1}{n}\sum_{i=1}^n X_i$}

\begin{center}
\setlength{\unitlength}{1cm}
\begin{picture}(5,7)
\put(-2,6){\framebox(9,1){Choose 5 Students from Class List with Replacement}}


\put(0.5,6){\vector(-1,-1){1.5}}
\put(-2.3,3.7){\framebox(2.5,0.65){Sample 1}}


\put(-1,3.5){\vector(0,-1){1}}
\put(-1.1,1.9){\makebox{\small $\bar{x}_1$}}

\pause

\put(2,6){\vector(0,-1){1.5}}
\put(0.7,3.7){\framebox(2.5,0.65){Sample 2}}


\put(2,3.5){\vector(0,-1){1}}
\put(1.9,1.9){\makebox{\small $\bar{x}_2$}}

\pause

\put(3.8,4){\makebox{...}}
\put(3.8,2){\makebox{...}}

\pause

\put(4.5,6){\vector(1,-1){1.5}}
\put(4.8,3.7){\framebox(2.5,0.65){Sample M}}


\put(6,3.5){\vector(0,-1){1}}
\put(5.9,1.9){\makebox{\small $\bar{x}_M$}}


\put(-1,1){\makebox{\small Repeat $M$ times $\rightarrow$  get $M$ different sample means}}

\pause

\put(-1.1,0.4){\makebox{\small \alert{Sampling Dist: relative frequencies of the $\bar{x}_i$ when $M = \infty$}}}

\end{picture}
\end{center}


\end{frame}
%%%%%%%%%%%%%%%%%%%%%%%%%%%%%%%%%%%%%%%%

\begin{frame}
\frametitle{Height of Econ 103 Students}
\begin{figure}
\includegraphics[scale = 0.4]{./images/height_hist}
\includegraphics[scale = 0.4]{./images/height_mean_n_5}
\caption{Left: Population, Right: Sampling distribution of $\bar{X}_5$}
\end{figure}
\end{frame}

%%%%%%%%%%%%%%%%%%%%%%%%%%%%%%%%%%%%%%%%
\begin{frame}
\frametitle{Histograms of sampling distribution of sample mean $\bar{X}_n$}
\alert{Random Sampling With Replacement, 10000 Reps. Each}
\begin{center}
\includegraphics[scale = 0.55]{./images/height_samples}
\end{center}
\end{frame}
%%%%%%%%%%%%%%%%%%%%%%%%%%%%%%%%%%%%%%%%
\begin{frame}
\frametitle{Population Distribution vs.\ Sampling Distribution of $\bar{X}_n$}

\begin{columns} 
\begin{column}[c]{6cm} 

 %FIRST COLUMN HERE 
\begin{figure}
\centering
\includegraphics[scale = 0.35]{./images/height_hist}
\end{figure}
\end{column} 
\begin{column}[c]{6cm} 

 %SECOND COLUMN HERE 
 \small
\begin{table}
\begin{tabular}{|rrr|}
\hline
&\multicolumn{2}{c|}{Sampling Dist.\ of $\bar{X}_n$}\\
$n$&Mean&Variance\\
\hline
5&67.6&3.6\\
10&67.5&1.8\\
20&67.5&0.8\\
50&67.5&0.2\\
\hline
\end{tabular}
\end{table}

\end{column} 
\end{columns} 
\begin{alertblock}{Things to Notice:}
\begin{enumerate}
	\item Sampling dist.\ ``correct on average'' 
	\item Sampling variability decreases with $n$
  \item Sampling dist.\ bell-shaped even though population isn't!
\end{enumerate}
\end{alertblock}
\end{frame}

%%%%%%%%%%%%%%%%%%%%%%%%%%%%%%%%%%%%%%%%
\section{Sampling Distributions}
%%%%%%%%%%%%%%%%%%%%%%%%%%%%%%%%%%%%%%%%
\begin{frame}
  \frametitle{Step 1: Population as RV rather than List of Objects}
\small
  \begin{tabular}[h]{cc}
    \hline
    \begin{minipage}[t]{0.6\textwidth}
      \begin{block}{Old Way}
       In the 2016 election, 65,853,625 out of 137,100,229 voters voted for Hillary Clinton\\
      \end{block}
    \end{minipage}
    &
    \begin{minipage}[t]{0.4\textwidth}
      \begin{alertblock}{New Way}
       Bernoulli$(p = 0.48)$ RV 
      \end{alertblock}
    \end{minipage} \\
    \hline
    \begin{minipage}[t]{0.6\textwidth}
      \begin{block}{Old Way}
        List of heights for 97 million US adult males with mean 69 in and std.\  dev.\ 6 in \\
      \end{block}
    \end{minipage}
    &
    \begin{minipage}[t]{0.4\textwidth}
      \begin{alertblock}{New Way}
        $N(\mu=69, \sigma^2 = 36)$ RV 
      \end{alertblock}
    \end{minipage} \\
    \hline
  \end{tabular}

  \vspace{1em}
  \alert{Second example assumes distribution of height is bell-shaped.}

\end{frame}
%%%%%%%%%%%%%%%%%%%%%%%%%%%%%%%%%%%%%%%%
\begin{frame}
  \frametitle{Step 2: iid RVs Represent Random Sampling from Popn.}
  \begin{block}{Hillary Voters Example}
   Poll random sample of 1000 people who voted in 2016:
   $$X_1, \hdots, X_{1000} \sim \mbox{ iid Bernoulli}(p = 0.48)$$
  \end{block}
  \begin{block}{Height Example}
   Measure the heights of random sample of 50 US males:
   $$Y_1, \hdots, Y_{50}  \sim \mbox{ iid } N(\mu = 69, \sigma^2 = 36)$$
  \end{block}

  \begin{block}{Key Question}
   What do the properties of the population imply about the properties of the sample? 
  \end{block}
\end{frame}
%%%%%%%%%%%%%%%%%%%%%%%%%%%%%%%%%%%%%%%%
\begin{frame}
  \frametitle{What does the population imply about the sample? \hfill\includegraphics[scale = 0.05]{./images/clicker}}
Suppose that exactly half of US voters chose Hillary Clinton in the 2016 election.
If you poll a random sample of 4 voters, what is the probability that \emph{exactly half} were Hillary supporters? 

\pause

\alert{$${4 \choose 2} \left( 1/2 \right)^2 \left( 1/2 \right)^2 = 3/8 = 0.375$$}
\end{frame}
%%%%%%%%%%%%%%%%%%%%%%%%%%%%%%%%%%%%%%%%
\begin{frame}
  \frametitle{The rest of the probabilities\dots}
  Suppose that exactly half of US voters plan to vote for Hillary Clinton and we poll a random sample of 4 voters.
  \begin{eqnarray*}
    P\left( \mbox{Exactly 0 Hillary Voters in the Sample} \right) &=& 0.0625\\
    P\left( \mbox{Exactly 1 Hillary Voters in the Sample} \right) &=& 0.25\\
    P\left( \mbox{Exactly 2 Hillary Voters in the Sample} \right) &=& 0.375\\
    P\left( \mbox{Exactly 3 Hillary Voters in the Sample} \right) &=& 0.25\\
    P\left( \mbox{Exactly 4 Hillary Voters in the Sample} \right) &=& 0.0625 
  \end{eqnarray*}

  \vspace{1em}
  \alert{You should be able to work these out yourself. If not, review the lecture slides on the Binomial RV.}
\end{frame}
%%%%%%%%%%%%%%%%%%%%%%%%%%%%%%%%%%%%%%%%
\begin{frame}
  \frametitle{Population Size is Irrelevant Under Random Sampling}

  \begin{block}{Crucial Point}
    \emph{None} of the preceding calculations involved the population size: I didn't even tell you what it was!
    We'll never talk about population size again in this course.
  \end{block}

  \begin{block}{Why?}
    Draw with replacement $\implies$ only the sample size and the \emph{proportion} of Hillary supporters in the population matter.
  \end{block}

\end{frame}
%%%%%%%%%%%%%%%%%%%%%%%%%%%%%%%%%%%%%%%%
\begin{frame}
  \frametitle{(Sample) Statistic}

  Any function of the data \emph{alone}, e.g.\ sample mean $\bar{x} = \frac{1}{n}\sum_{i=1}^n x_i$. 
  Used to estimate a population parameter: e.g.\ $\bar{x}$ estimates of
  $\mu$.

\end{frame}

%%%%%%%%%%%%%%%%%%%%%%%%%%%%%%%%%%%%%%%%
\begin{frame}
  \frametitle{Step 3: Random Sampling $\Rightarrow$ \emph{Sample Statistics} are RVs} 

  \alert{This is \emph{the crucial point of the course}: if we draw a random sample, the dataset we get is random. Since a statistic is a function of the data, it is a random variable!} 
\end{frame}
%%%%%%%%%%%%%%%%%%%%%%%%%%%%%%%%%%%%%%%%
%\begin{frame}
%  \frametitle{A Sample Statistic in the Polling Example \hfill\includegraphics[scale = 0.05]{./images/clicker}}
%  Suppose that exactly half of voters in the population support Hillary Clinton and we poll a random sample of 4 voters. If we code Hillary supporters as ``1'' and everyone else as ``0'' then what are the possible values of the sample mean in our dataset?
%
%  \vspace{1em}
%  \begin{enumerate}[(a)]
%    \item $(0,1)$
%    \item $\left\{ 0, 0.25, 0.5, 0.75, 1 \right\}$
%    \item $\left\{ 0,1,2,3,4 \right\}$
%    \item $(-\infty, \infty)$
%    \item Not enough information to determine.
%  \end{enumerate}
%
%\end{frame}
%%%%%%%%%%%%%%%%%%%%%%%%%%%%%%%%%%%%%%%%
\begin{frame}
  \frametitle{Sampling Distribution}
  Under random sampling, a statistic is a RV so it has a PDF if continuous or PMF if discrete: this is its \alert{sampling distribution}. 

  \begin{block}{Sampling Dist.\ of Sample Mean in Polling Example}
   \begin{eqnarray*}
   p(0) &=&  0.0625\\
   p(0.25) &=&  0.25\\ 
   p(0.5) &=&  0.375\\
   p(0.75)&=& 0.25\\ 
   p(1) &=&  0.0625
   \end{eqnarray*}

  \end{block}
\end{frame}
%%%%%%%%%%%%%%%%%%%%%%%%%%%%%%%%%%%%%%%%
\section{Estimator versus Estimate}
%%%%%%%%%%%%%%%%%%%%%%%%%%%%%%%%%%%%%%%%
\begin{frame}
  \frametitle{Contradiction? No, but we need better terminology\dots}
  \begin{itemize}
    \item Under random sampling, a statistic is a RV
    \item Given dataset is \emph{fixed} so statistic is a \emph{constant number}
    \item Distinguish between: \alert{Estimator} vs.\ \alert{Estimate}
  \end{itemize}

  \begin{alertblock}{Estimator}
   Description of a general procedure. 
  \end{alertblock}
  \begin{alertblock}{Estimate}
    Particular result obtained from applying the procedure.
  \end{alertblock}
\end{frame}
%%%%%%%%%%%%%%%%%%%%%%%%%%%%%%%%%%%%%%%%
\begin{frame}
  \begin{block}{$\bar{X}_n$ is an Estimator = Procedure = Random Variable}
\begin{enumerate}
\item Take a random sample: $X_1, \hdots, X_n \quad$ 
\item Average what you get: $\bar{X}_n = \frac{1}{n}\sum_{i=1}^n X_i\quad$ 
\end{enumerate}
\end{block}

\pause
\begin{block}{$\bar{x}$ is an Estimate = Result of Procedure = Constant}
 \begin{itemize}
\item Result of taking a random sample was the dataset: $x_1, \hdots, x_n$ 
\item Result of averaging the observed data was $\bar{x} = \frac{1}{n}\sum_{i=1}^n x_i$ 
\end{itemize}
\end{block}

\pause
\begin{block}{Sampling Distribution of $\bar{X}_n$}
  \alert{Thought experiment:} suppose I were to repeat the procedure of taking the mean of a random sample over and over \alert{forever}. What \alert{relative frequencies} would I get for the sample means?
\end{block}

\end{frame}


%%%%%%%%%%%%%%%%%%%%%%%%%%%%%%%%%%%%%%%%
\section{Unbiasedness of Sample Mean}
%%%%%%%%%%%%%%%%%%%%%%%%%%%%%%%%%%%%%%%%
%\begin{frame}
%\frametitle{$X_1,\hdots, X_{9} \sim \mbox{ iid }$ with $\mu=5$, $\sigma^2 =36$. \hfill\includegraphics[scale = 0.05]{./images/clicker}}
%
%\large Calculate:
%	 $$E(\bar{X}) = E\left[\frac{1}{9}(X_1 + X_2 + \hdots + X_{9})\right]$$
%\end{frame}
%%%%%%%%%%%%%%%%%%%%%%%%%%%%%%%%%%%%%%%%
\begin{frame}
\frametitle{Mean of Sampling Distribution of $\bar{X}_n$}
\alert{$X_1, \hdots, X_n \sim \mbox{iid with mean }\mu$}
\begin{eqnarray*}
E[\bar{X_n}] = E\left[ \frac{1}{n}\sum_{i=1}^n X_i \right]\pause = \frac{1}{n} \sum_{i=1}^n E[X_i] = \pause\frac{1}{n} \sum_{i=1}^n \mu = \pause \frac{n\mu}{n} = \mu
\end{eqnarray*}
\alert{Hence, sample mean is ``correct on average.'' The formal term for this is \emph{unbiased}.}
\end{frame}

%%%%%%%%%%%%%%%%%%%%%%%%%%%%%%%%%%%%%%%%
\section{Standard Error of the Mean}
%%%%%%%%%%%%%%%%%%%%%%%%%%%%%%%%%%%%%%%%
%\begin{frame}
%\frametitle{$X_1,\hdots, X_{9} \sim \mbox{ iid }$ with $\mu=5$, $\sigma^2 = 36$. \hfill\includegraphics[scale = 0.05]{./images/clicker}}
%
%\large Calculate:
%	 $$Var(\bar{X}) = Var\left[\frac{1}{9}(X_1 + X_2 + \hdots + X_{9})\right]$$
%\end{frame}
%%%%%%%%%%%%%%%%%%%%%%%%%%%%%%%%%%%%%%%%

\begin{frame}
\frametitle{Variance of Sampling Distribution of $\bar{X}_n$}
\alert{$X_1, \hdots, X_n \sim \mbox{iid with mean }\mu \mbox{ and variance } \sigma^2$}
\begin{eqnarray*}
Var[\bar{X_n}] &=& Var\left[ \frac{1}{n}\sum_{i=1}^n X_i \right] \pause= \frac{1}{n^2} \sum_{i=1}^n Var(X_i) \\
&=& \pause \frac{1}{n^2} \sum_{i=1}^n \sigma^2 = \pause \frac{n\sigma^2}{n^2} =  \frac{\sigma^2}{n}
\end{eqnarray*}

\alert{Hence the variance of the sample mean \emph{decreases linearly with sample size}.}
\end{frame}
%%%%%%%%%%%%%%%%%%%%%%%%%%%%%%%%%%%%%%%%
\begin{frame}
\frametitle{Standard Error}
Std. Dev.\ of estimator's sampling dist.\ is called \alert{standard error}.
\begin{block}{Standard Error of the Sample Mean}
$SE(\bar{X}_n)= \sqrt{Var\left(\bar{X}_n\right)}= \sqrt{\sigma^2/n}=\sigma/\sqrt{n}$
\end{block}
\end{frame}

%%%%%%%%%%%%%%%%%%%%%%%%%%%%%%%%%%%%%%%%
%\begin{frame}
%\frametitle{More Generally and More Formally:}
%\framesubtitle{Sample mean isn't the only thing with a sampling distribution!}
%\begin{block}{Estimator}
%  A function $T(X_1, \hdots, X_n)$ of the RVs that represent the \emph{procedure} of drawing a random sample, hence a RV itself.
%\end{block}
%
%\begin{block}{Sampling Distribution}
%The probability distribution (PMF of PDF) of an Estimator. 
%\end{block}
%
%\begin{block}{Estimate}
%A function $T(x_1, \hdots, x_n)$ of the \emph{observed data}, i.e.\ the \emph{realizations} of the random variables we use to represent random sampling. 
%Since its a function of constants, an estimate is itself a constant.
%\end{block}
%
%\end{frame}
%
%%%%%%%%%%%%%%%%%%%%%%%%%%%%%%%%%%%%%%%%%
%
%
%\begin{frame}
%
%\begin{center}
%\setlength{\unitlength}{1cm}
%\begin{picture}(5,7)
%\put(1,6){\framebox(3,1){Population: $f(x)$}}
%
%\put(4.5,6.5){\makebox{\small \alert{\emph{Probability Distribution}}}}
%\put(2.5,6){\vector(0,-1){1}}
%\put(0,3){\framebox(5,1){$X_1, X_2, \hdots, X_n \sim \mbox{iid } f(x)$}}
%\put(0.5,4.5){\makebox{\small Random Sample of Size $n$}}
%
%\put(-3,3.4){\makebox{\small \alert{\emph{Random Variables}}}}
%
%\put(0.5,3){\vector(-1,-1){1.5}}
%\put(-2.3,0.7){\framebox(2.5,0.65){$x_1^{(1)}, \hdots, x_n^{(1)}$}}
%
%\put(2,3){\vector(0,-1){1.5}}
%\put(0.7,0.7){\framebox(2.5,0.65){$x_1^{(2)}, \hdots, x_n^{(2)}$}}
%
%
%\put(3.5,2){\makebox{...}}
%
%
%\put(4.5,3){\vector(1,-1){1.5}}
%\put(4.8,0.7){\framebox(2.5,0.65){$x_1^{(M)}, \hdots, x_n^{(M)}$}}
%
%\put(-1,-0.2){\makebox{\small $M$ Replications, each containing $n$ Observations}}
%
%\put(5.6,2.6){\makebox{\small \alert{\emph{Realizations}}}}
%\put(5.6,2.2){\makebox{\small \alert{\emph{(Constants)}}}}
%
%\end{picture}
%\end{center}
%
%
%\end{frame}
%%%%%%%%%%%%%%%%%%%%%%%%%%%%%%%%%%%%%%%%%
%
%\begin{frame}
%
%\begin{center}
%\setlength{\unitlength}{1cm}
%\begin{picture}(5,7)
%\put(-1,5){\framebox(4.5,0.6){$X_1, X_2, \hdots, X_n \sim \mbox{iid } f(x)$}}
%\put(-0.7,6){\makebox{\small Random Sample of Size $n$}}
%
%
%\put(-2.7,5.6){\makebox{\small \alert{\emph{Random}}}}
%\put(-2.7,5.2){\makebox{\small \alert{\emph{Variables}}}}
%
%
%\put(3.8,5.3){\vector(1,0){1}}
%\put(5.5,6){\makebox{\small Estimator}}
%\put(5,5){\framebox(2.5,0.6){$T(X_1,\hdots, X_n)$}}
%
%
%
%\put(5,4){\makebox{\small \alert{$\displaystyle\bar{X}_n= \frac{1}{n} \sum_{i=1}^n X_i$}}}
%
%
%
%
%
%\put(1.2,4.7){\vector(0,-1){2.3}}
%\put(-0.2,1.7){\makebox{\small Observations (Data)}}
%\put(0,0.7){\framebox(2.5,0.6){$x_1, \hdots, x_n$}}
%\put(-2.7,0.9){\makebox{\small \alert{\emph{Constants}}}}
%
%
%\put(4.5,1.7){\makebox{\small Estimate}}
%\put(4,0.7){\framebox(2.5,0.6){$T(x_1, \hdots, x_n)$}}
%\put(2.8,1){\vector(1,0){1}}
%\put(4,-0.2){\makebox{\small \alert{$\displaystyle\bar{x}= \frac{1}{n} \sum_{i=1}^n x_i$}}}
%
%\end{picture}
%\end{center}
%
%
%\end{frame}
%%%%%%%%%%%%%%%%%%%%%%%%%%%%%%%%%%%%%%%%%
