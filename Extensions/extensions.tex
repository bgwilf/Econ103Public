\documentclass[addpoints,12pt]{exam}
\usepackage{amsmath, amssymb}
\linespread{1.1}
\usepackage{hyperref}

\printanswers
%\noprintanswers

\title{Extension Problems}
\author{Econ 103}
\date{Spring 2018}

\begin{document}
\maketitle

\section*{About This Document}


%%%%%%%%%%%%%%%%%%%%%%%%%%%%%%%%%%%%%%%%%%%%%%%%%%%%%%%%%%%%%%%%%%

\fullwidth{\section*{Lecture \#1}}
\begin{questions}

\question A long time ago, the graduate school at a famous university admitted 4000 of their 8000 male applicants versus 1500 of their 4500 female applicants. 
	\begin{parts}
		\part Calculate the difference in admission rates between men and women. What does your calculation suggest?
		\begin{solution}[1.5in]
		The rate for men is $4000/8000 =50\%$ while that for women is $1500/4500 \approx 33\%$ so the difference is 17\%. It appears that women are less likely to be accepted to the graduate school.
		\end{solution}
		\part To get a better sense of the situation, some researchers broke these data down by area of study. Here is what they found:
			\begin{center}
			\begin{tabular}{lcc|cc}
				&\multicolumn{2}{c|}{Men}&\multicolumn{2}{c}{Women}\\
				&\# Applicants & \# Admitted & \# Applicants& \# Admitted\\
				\hline
				Arts &2000 & 400 & 3600 & 900\\
				Sciences& 6000 & 3600 & 900 & 600\\
				\hline
				Totals&8000 & 4000 & 4500 & 1500\\
				\end{tabular}
			\end{center}
			
			Calculate the difference in admissions rates for men and women studying Arts. Do the same for Sciences.
			\begin{solution}[1.75in]
			For Arts, the admission rate is $400/2000=20\%$ for men versus $900/3600 = 25\%$ for women. For Sciences $3600/6000 = 60\%$ for men versus $600/900  \approx 67\%$ for women. In summary:
				\begin{center}
				\begin{tabular}{l|ccc}
				&Men&Women&Difference\\
				\hline
				Arts&20\% & 25\%& -5\% \\
				Sciences& 60\% & 67\% &  -7\% \\
				\hline
				Overall&50\% & 33\% & 17\%
				\end{tabular}
				\end{center}
			\end{solution}
		\part Compare your results from part (a) to part (b). Explain the discrepancy using what you know about observational studies.
			\begin{solution}[1.75in]
				When we compare overall rates, women are less likely to be admitted than men. In each field of study, however, women are \emph{more} likely to be admitted. In this example, field of study is a \emph{confounder}: women are disproportionately applying to study Arts and Arts have much lower admissions rates than Sciences.
			\end{solution}
	\end{parts}

%%%%%%%%%%%%%%%%%%%%%%%%%%%%%%%%%%%%%%%%%%%%%%%%%%%%%%%%%%%%%%%%%%

\fullwidth{\section*{Lecture \#2}}
	
\question The \emph{mean deviation} is a measure of dispersion that we did not cover in class. It is defined as follows:
	$$MD = \frac{1}{n}\sum_{i=1}^n |x_i - \bar{x}|$$
	\begin{parts}
		\part Explain why this formula averages the absolute value of deviations from the mean rather than the deviations themselves.
		\begin{solution}
		As we showed in class, the average deviation from the sample mean is zero regardless of the dataset. Taking the absolute value is similar to squaring the deviations: it makes sure that the positive ones don't cancel out the negative ones.
		\end{solution}
		\part Which would you expect to be more sensitive to outliers: the mean deviation or the variance? Explain.
		\begin{solution}
		The variance is calculated from squared deviations. When $x$ is far from zero, $x^2$ is much larger than $|x|$ so large deviations ``count more'' when calculating the variance. Thus, the variance will be more sensitive to outliers. 
		\end{solution}
	\end{parts}

\question Let $m$ be a constant and $x_1, \dots, x_n$ be an observed dataset.
  \begin{parts} 
    \part Show that $\displaystyle \sum_{i=1}^n (x_i - m)^2 = \sum_{i=1}^n x_i^2 - 2m \sum_{i=1}^n x_i + nm^2$.
\begin{solution}
  \begin{align*}
    \displaystyle \sum_{i=1}^n (x_i - m)^2 &= \sum_{i=1}^n (x_i^2 - 2mx_i + m^2)\\
    &= \sum_{i=1}^n x_i^2 - \sum_{i=1}^n 2mx_i + \sum_{i=1}^n m^2\\
    &= \sum_{i=1}^n x_i^2 - 2m\sum_{i=1}^n x_i + nm^2
\end{align*}
\end{solution}
  \part Using the preceding part, show that $\displaystyle\sum_{i=1}^n (x_i - \bar{x})^2 = \sum_{i=1}^n x_i^2 - n\bar{x}^2$.
  \begin{solution}
    Solving this requires two observations.
    First, note that $\bar{x}$ is a \emph{constant}, i.e.\ that it does not have an index of summation.
    Second, note that $\sum_{i=1}^n x_i = n\bar{x}$.
    Hence, taking $m = \bar{x}$ in the formula from the preceding part,
    \begin{align*} 
      \displaystyle\sum_{i=1}^n (x_i - \bar{x})^2 &= \sum_{i=1}^n x_i^2 - 2\bar{x} \sum_{i=1}^n x_i + n\bar{x}^2\\
      &= \sum_{i=1}^n x_i^2 - 2\bar{x} (n\bar{x}) + n\bar{x}^2\\
      &= \sum_{i=1}^n x_i^2 - 2n\bar{x}^2  + n\bar{x}^2\\
      &= \sum_{i=1}^n x_i^2 - n\bar{x}^2\\
\end{align*}
  \end{solution}
\end{parts}

%%%%%%%%%%%%%%%%%%%%%%%%%%%%%%%%%%%%%%%%%%%%%%%%%%%%%%%%%%%%%%%%%%

\fullwidth{\section*{Lecture \#3}}

\question Consider a dataset $x_1, \hdots, x_n$. Suppose I multiply each observation by a constant $d$ and then add another constant $c$, so that $x_i$ is replaced by $c + dx_i$.
	\begin{parts}
		\part How does this change the sample mean? Prove your answer.
			\begin{solution}
				\begin{eqnarray*}
					\frac{1}{n} \sum_{i=1}^n (c + dx_i)&=&\frac{1}{n} \sum_{i=1}^n c + d \left(\frac{1}{n} \sum_{i=1}^n x_i\right) = c + d\bar{x}
				\end{eqnarray*}
			\end{solution}
		\part How does this change the sample variance? Prove your answer.
		\begin{solution}
			$$\frac{1}{n-1} \sum_{i=1}^n [(c + dx_i) - (c + d\bar{x})]^2 = \frac{1}{n-1} \sum_{i=1}^n [d(x_i - \bar{x})]^2 = d^2 s_x^2$$
		\end{solution}
		\part How does this change the sample standard deviation? Prove your answer.
			\begin{solution}
			The new standard deviation is $|d| s_x$, the positive square root of the variance.
			\end{solution}
		\part How does this change the sample z-scores? Prove your answer.
			\begin{solution}
			They are unchanged as long as $d$ is positive, but the sign will flip if $d$ is negative:
				$$\frac{(c + d x_i) - (c + d\bar{x})}{d s_x} = \frac{d (x_i - \bar{x})}{d s_x} = \frac{x_i - \bar{x}}{s_x}$$
			\end{solution}
	\end{parts}

%%%%%%%%%%%%%%%%%%%%%%%%%%%%%%%%%%%%%%%%%%%%%%%%%%%%%%%%%%%%%%%%%
\fullwidth{\section*{Lecture \#4}}

  \question Define the z-scores 
  \[w_i = \frac{x_i - \bar{x}}{s_x}, \;\;\mbox{and }\;\; z_i = \frac{y_i - \bar{y}}{s_y}.\]
  Show that if we carry out a regression with $z_i$ in place of $y_i$ and $w_i$ in place of $x_i$, the intercept $a^*$ will be zero while the slope $b^*$ will be $r_{xy}$, the correlation between $x$ and $y$.
    \begin{solution}
     All we need to do is replace $x_i$ with $w_i$ and $y_i$ with $z_i$ in the formulas we already derived for the regression slope and intercept:
     		$$\begin{array}{lr}a = \bar{y} - b\bar{x}, & b = \displaystyle \frac{s_{xy}}{s_x^2}\end{array}$$
     		And use the properties of z-scores from class. Let $a^*$ be the intercept for the regression with z-scores, and $b^*$ be the corresponding slope. We have:
        $$a^* = \bar{z} - b^* \bar{w} = 0$$
     	 	since the mean of the z-scores is zero, as we showed in class. To find the slope, we need to covariance between the z-scores, and the variance of the z-scores for $x$:
     	 		$$b^* = \frac{s_{wz}}{s_{w}^2}$$
     	 		But since sample variance of z-scores is always one, $b^* = s_{wz}$. Now, by the definition of the sample covariance, the fact that the mean of z-scores is zero, and the definition of a z-score:
     	 		\begin{align*}
     	 		s_{wz} &= \frac{1}{n-1} \sum_{i=1}^n (w - \bar{w})(z - \bar{z}) = \frac{1}{n-1} \sum_{i=1}^n z_{x_i}z_{y_i}\\
     	 		&= \frac{1}{n-1} \sum_{i=1}^n \left(\frac{x_i - \bar{x}}{s_x}\right)\left(\frac{y_i - \bar{y}}{s_y}\right)\\
     	 		&= r_{xy}
     	 		\end{align*}
    \end{solution}

  \question This question concerns a phenomenon called \emph{regression to the mean}. Before attempting this problem, read Chapter 17 of \emph{Thinking Fast and Slow} by Kahneman.
  \begin{parts}
    \part Lothario, an unscrupulous economics major, runs the following scam. After the first midterm of Econ 103 he seeks out the students who did extremely poorly and offers to sell them ``statistics pills.'' He promises that if they take the pills before the second midterm, their scores will improve. The pills are, in fact, M\&Ms and don't actually improve one's performance on statistics exams. The overwhelming majority of Lothario's former customers, however, swear that the pills really work: their scores improved on the second midterm. What's your explanation?
  \begin{solution}
This is an example of regression to the mean. The students Lothario seeks out were both unprepared for the midterm \emph{and} got unlucky: the correlation between exam scores is less than one. It is very unlikely that they will be unlucky twice in a row, so their performance on the second exam will almost certainly be higher. Our best guess of their second score is closer to the mean than their first score.
  \end{solution}
		\part Let $\hat{y}$ denote our prediction of $y$ from a linear regression model: $\hat{y} = a + b x$ and let $r$ be the correlation coefficient between $x$ and $y$. Show that $$\frac{\hat{y} - \bar{y}}{s_y} = r \left( \frac{x - \bar{x}}{s_x}\right)$$
			\begin{solution}
				\begin{eqnarray*}
					\hat{y} &=& a + bx \\
					\hat{y}&=& (\bar{y} - b \bar{x}) + bx\\
					\hat{y} - \bar{y} &=& b(x - \bar{x})\\
					\hat{y} - \bar{y} &=& \frac{s_{xy}}{s_x^2}(x - \bar{x})\\
					\hat{y} - \bar{y} &=& \frac{s_{xy}}{s_x}\left(\frac{x - \bar{x}}{s_x}\right)\\
					\frac{\hat{y} - \bar{y}}{s_y} &=& \frac{s_{xy}}{s_x s_y}\left(\frac{x - \bar{x}}{s_x}\right)\\
										\frac{\hat{y} - \bar{y}}{s_y} &=& r\left(\frac{x - \bar{x}}{s_x}\right)
				\end{eqnarray*}
			\end{solution}
		\part Using the equation derived in (d), briefly explain``regression to the mean.''
			\begin{solution}
			The formula shows that unless $r$ is one or negative one, perfect positive or negative correlation, our best linear prediction of $y$ based on knowledge given $x$ is closer to the mean of the $y$-observations (relative to the standard deviation of the $y$-observations) than $x$ is to mean of the $x$-observations (relative to the standard deviation of the $x$-observations). If $x$ is very large, for example, we would predict that $y$ will be large too, but not as large.
			\end{solution}
	\end{parts}

%%%%%%%%%%%%%%%%%%%%%%%%%%%%%%%%%%%%%%%%%%%%%%%%%%%%%%%%%%%%%%%%%
\fullwidth{\section*{No extension problems for Lecture \#5}}
	
%%%%%%%%%%%%%%%%%%%%%%%%%%%%%%%%%%%%%%%%%%%%%%%%%%%%%%%%%%%%%%%%%
\fullwidth{\section*{Lecture \#6}}

\end{questions}


\end{document}
